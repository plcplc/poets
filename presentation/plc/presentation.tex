% $Header: /cvsroot/latex-beamer/latex-beamer/solutions/generic-talks/generic-ornate-15min-45min.en.tex,v 1.5 2007/01/28 20:48:23 tantau Exp $

\documentclass{beamer}

% This file is a solution template for:

% - Giving a talk on some subject.
% - The talk is between 15min and 45min long.
% - Style is ornate.



% Copyright 2004 by Till Tantau <tantau@users.sourceforge.net>.
%
% In principle, this file can be redistributed and/or modified under
% the terms of the GNU Public License, version 2.
%
% However, this file is supposed to be a template to be modified
% for your own needs. For this reason, if you use this file as a
% template and not specifically distribute it as part of a another
% package/program, I grant the extra permission to freely copy and
% modify this file as you see fit and even to delete this copyright
% notice.

\usepackage{verbatim}
\usepackage{amsmath}
\usepackage{microtype} % ++
%\mathlig{||}{\ |\ }
\usepackage{listings}
%\newcommand{\kw}[1]{\text{\lstinline{#1}}}
\newcommand{\kw}[1]{\mathbf{#1}}

\mode<presentation>
{
  \usetheme{Warsaw}
  % or ...

  \setbeamercovered{transparent}
  % or whatever (possibly just delete it)
}


\usepackage[english]{babel}
% or whatever

\usepackage[latin1]{inputenc}
% or whatever

\usepackage{times}
\usepackage[T1]{fontenc}
% Or whatever. Note that the encoding and the font should match. If T1
% does not look nice, try deleting the line with the fontenc.


\title % (optional, use only with long paper titles)
{CSL predicate extensions}

\subtitle
{Report presentation} % (optional)

\author % (optional, use only with lots of authors)
{Philip Lykke Carlsen}
% - Use the \inst{?} command only if the authors have different
%   affiliation.

%\institute[Universities of Somewhere and Elsewhere] % (optional, but mostly needed)
%{
  %\inst{1}%
  %Department of Computer Science\\
  %University of Somewhere
  %\and
  %\inst{2}%
  %Department of Theoretical Philosophy\\
  %University of Elsewhere}
% - Use the \inst command only if there are several affiliations.
% - Keep it simple, no one is interested in your street address.

\date[Exam] % (optional)
{ 3. Sep 2010 / Examination}

%\subject{Talks}
% This is only inserted into the PDF information catalog. Can be left
% out.



% If you have a file called "university-logo-filename.xxx", where xxx
% is a graphic format that can be processed by latex or pdflatex,
% resp., then you can add a logo as follows:

% \pgfdeclareimage[height=0.5cm]{university-logo}{university-logo-filename}
% \logo{\pgfuseimage{university-logo}}



% Delete this, if you do not want the table of contents to pop up at
% the beginning of each subsection:
%\AtBeginSubsection[]
%{
%  \begin{frame}<beamer>{Outline}
%    \tableofcontents[currentsection,currentsubsection]
%  \end{frame}
%}


% If you wish to uncover everything in a step-wise fashion, uncomment
% the following command:

%\beamerdefaultoverlayspecification{<+->}


\begin{document}

\begin{frame}
  \titlepage
\end{frame}

\begin{frame}{Outline}
  \tableofcontents
  % You might wish to add the option [pausesections]
\end{frame}


% Since this a solution template for a generic talk, very little can
% be said about how it should be structured. However, the talk length
% of between 15min and 45min and the theme suggest that you stick to
% the following rules:

% - Exactly two or three sections (other than the summary).
% - At *most* three subsections per section.
% - Talk about 30s to 2min per frame. So there should be between about
%   15 and 30 frames, all told.

\section{Motivation}

%\subsection[Short First Subsection Name]{First Subsection Name}

\begin{frame}{Motivation}
  % - A title should summarize the slide in an understandable fashion
  %   for anyone how does not follow everything on the slide itself.

  \begin{itemize}
  \item
    Only predifined functions (and only few over lists)
  \item
    Low expressive power.
  \end{itemize}
\end{frame}

\section{Solution}

\begin{frame}{Solution}

  \begin{itemize}
    \item Allow function definitions.
    \item \alert{Restrict recursion}, as non-terminating predicates voids a contract of meaning.
  \end{itemize}

\end{frame}

\begin{frame}{Extension points}
CSL needed extensions in
\begin{itemize}
  \item The Abstract Syntax
  \item The Parser
  \item The Evaluator
  \item The Typeinferer
\end{itemize}
\end{frame}

\subsection*{Syntax}

\begin{frame}{Syntax Extensions}
  \begin{align*}%[box=\widefbox]
    p ::= &~ \ldots || \kw{val}~ id : type = e \tag{CSL statements} \\
       & \quad\quad || \kw{fun}~ id : t~ (id_1 : t_1, id_2 : t_2, \ldots, id_n : t_n) = e  \\
    e ::= &~ \ldots || \lambda id_1~ id_2~ \ldots~ id_n -> e || e_1\ e_2  \tag{CSL predicates}\\
       & \quad\quad || \kw{if}~ c~ \kw{then}~ e_1~ \kw{else}~ e_2 \\
    t ::= &~ \ldots || type_1 -> type_2 ||
    typevar \tag{CSL types}
  \end{align*}
\end{frame}

\begin{frame}[fragile]{Abstract Syntax}

\begin{verbatim}
-- |Value binding.
data ValueBinding = ValueBinding{
      bindingType  :: CType, -- ^Type
      bindingName  :: Var, -- ^Variable name of binding.
      bindingValue :: CExpr -- ^Value to be bound.
    } deriving (Typeable)
\end{verbatim}

\end{frame}

\begin{frame}[fragile]{Abstract Syntax}

\begin{verbatim}
-- |AST for extended expressions in CSL.
data Exp e =
  ...
  -- |Anonymous function, a Lambda.
  | ELambda{
        expLArg     :: Var -- ^Argument
      , expLBody    :: e   -- ^Function body
      }
  -- |Conditional expression.
  | EIfThenElse{
        expCondition :: e -- ^Condition.
      , expThen :: e      -- ^Then branch.
      , expElse :: e      -- ^Else branch.
      }
\end{verbatim}
\end{frame}

\subsection*{Evaluator}

\begin{frame}{Evaluator Algebra}
\begin{itemize}
\item Based on variable environment rather than substitution.
\item \dots Which means that the algebra produces monadic actions that yield a
value when parametrised with an environment, rather than values directly.
\item Extended to include hardcoded versions of \textbf{foldl},\textbf{foldr} and \textbf{ceil}.
\end{itemize}
\end{frame}

\begin{frame}{Evaluator Algebra}

\begin{itemize}
  \item Functions represented as haskell-functions
  \item Value datatype had to be encapsulated in a newtype (CValue)
\end{itemize}

\end{frame}

\subsection*{Type inference}

\begin{frame}{Typeinferer Algebra}
Added constraints for:
\begin{itemize}
  \item builtin functions
  \item lambda functions
\end{itemize}

\end{frame}

\section{Conclusion}

\begin{frame}{Conclusion}

\begin{itemize}
  \item The basic most wanted features have been implemented
  \item parserfejl?
\end{itemize}

\end{frame}
\end{document}


