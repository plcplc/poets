\documentclass[10pt,a4paper,final,oneside,openany,article]{memoir}
\usepackage[utf8]{inputenc}
\usepackage[danish, british]{babel}
%\usepackage{hyperref}
\usepackage{graphicx}
\usepackage{url}
\usepackage{listings}
\usepackage[draft]{fixme}
\usepackage{amsmath}
\usepackage{semantic}

% Fonts configuration
%  - Palatino and Bitstream Vera Sans Mono for verbatim
\usepackage[T1]{fontenc}
\usepackage{palatino}
\usepackage[sc]{mathpazo} % Math font for Palatino
\usepackage{bera}
\linespread{1.05} % Palatino needs more leading (space between lines)
\usepackage{microtype} % ++

% Commands
\newcommand{\twofig}[6]{
\begin{figure}[htp!]
\begin{center}
\fbox{    \subfigure[#1]{    \includegraphics[width=0.4\textwidth]{#2}}
    \subfigure[#3]{    \includegraphics[width=0.4\textwidth]{#4}    } }
\caption{#5}
\label{#6}
\end{center}
\end{figure}
}

\newcommand{\onefig}[4]{
\begin{figure}[htp!]
\begin{center}
\fbox{    \includegraphics[width= #4 \textwidth]{#1}}
\caption{#2}
\label{#3}
\end{center}
\end{figure}
}

\mathlig{||}{\ |\ }

% Bibliography
\usepackage[style=alphabetic,natbib=true]{biblatex}
\bibliography{../bibliography}


% Language syntax highlighting
\lstdefinelanguage{CSL}{ keywords={contract, true, false, if, then,
    else, where, due, immediately, within, after, or, and, andalso,
    empty},
  keywordstyle = \textbf,
  commentstyle = \textit,
  stringstyle = \ttfamily,
  basicstyle   = \small\ttfamily,
  numberstyle=\tiny,
  sensitive = true,
  morecomment=[s]{/*}{*/},
  morecomment = [l]{//}
}
\lstset{language = CSL}

% Document meta
\title{Extending the POETS \\ Contract Specification Language\\
       \vspace{0.1cm}
        \small{A POETS Coding Extravaganza}}
\author{Brian Søborg Mathiasen (soborg@diku.dk) \and
        Philip Lykke Carlsen (plcplc@gmail.com) \and
        Martin Dybdal (dybber@dybber.dk)}
\date{27th August 2010}

% Layouting
\setcounter{tocdepth}{3}
\pagestyle{plain}
\setcounter{secnumdepth}{1}

\begin{document}
\maketitle

\begin{abstract}
  As part of the 3gERP and POETS projects [reference], there has been
  developed a Contract Specification Language called CSL, which makes
  it possible to write down sets of commitments between parties
  (e.g. individuals and companies) and combining those commitments
  into legally binding agreements, contracts. Although it overcomes a
  number of limitations in earlier contract languages, it isn't as
  feature complete as is expected from modern languages. For instance,
  it supports lists, but has no primitives for manipulating them and
  it provides no means for defining your own function primitives and
  are thus limited to a small set of built in functions.

  This report is written as part of the course "`POETS Summer of
  Code"' and presents a number of extensions to the existing language
  on basis of a number of contract schemes of practical use, that were
  impossible to formulate with the existing language. The extensions
  ensure the integrity of the language core features described in
  \cite{hvitved10}.
\end{abstract}

\listoffixmes

\fxnote{Kalder vi det expression language eller predicate language?}

\newpage
\chapter{Introduction}
% Har vi brug for hele Toms motivation? Det virker meget som en stor
% gentagelse, når man har læst introduktionen til hans artikel eller
% techreport.
Recently, Tom Hvitved [citation], presented a language for specifying
multi-party contracts between individuals or companies, for use in third
generation ERP systems. The motivation of such a language is based on
research [citation] stating that contract lifecycle management (CLM) has
become a critical key to success for enterprises. Current
state-of-the-art enterprise systems such as Microsoft Dynamics NAV do
not represent contracts explicitly as first-class objects, but rather
via low-level code and database schemes. Such contracts are hard to
maintain, costly and time consuming, and analyses are difficult to
perform. The proposed language overcomes these difficulties and
introduce several features, such as absolute temporal constraints
(deadlines), relative temporal constraints (sequential ordering),
contrary-to-duty clauses, run-time monitoring, blame assignment and much
more [citation].

However, despite these many features of the language, it is far from
complete. In fact, several key features used in real life contract
management are impossible to define in the existing language. In this
paper, we will introduce a number of extensions to the existing contract
language, as well as define an extension to the expression language
usable within contract templates, in order to provide a richer array of
possibilities within the system. We will also ensure, that any and all
pre-existing features of the system will co-exist with the extension,
and that the semantics of all semantics remain sound and in line with
the initial intentions of the language (??).

To further highlight the need of these extensions, we will provide a
number of examples usable in the real world, that are impossible to
formalize in the existing system.

\fxwarning{Forklaring af clause templates og contract templates}

\section{Motivating cases}
\label{motivation}
% Case navnene skal måske omskrives så de antyder problemer frem for
% løsninger, alternativt skal sektionen hede "limitations" frem for at
% være motiverende.

\paragraph{Case 1: Functions} The current language doesn't provide
anyway of writing functions for the expression level. Clauses
templates are on the clause-level. Functions are important
abstractions in all modern languages and without them code maintenance
can become complicated, if an expression has to be duplicated instead
of defined once by a function.

\paragraph{Case 2: List manipulation.} In the current language it is
impossible to manipulate the contents of lists.
% Example contract where lists needs to be manipulated

\paragraph{Case 3: Conditional expression.} Decisions can't be taken
on the expression level, only decisions that selects between several
clauses can be taken and thus an amount of repetition would be needed
without expression level conditionals.


\chapter{Analysing requirements}
In this section we analyse how we best can make it possible to solve
the cases mentioned in Section \ref{motivation}.

\paragraph{Case 1: Functions} As CSL is a declarative language, our
functions should be inspired by declarative principles. We therefore
draw from the world of functional programming and define functions as
lambda expressions that acts as primitive values, which can be passed
between other functions (higher-order functions). An important
limitation for expressing functions on the expression level is that we
do now allow recursive functions. The important rationale for this
decision is termination. By allowing recursive functions, we can never
be sure that they terminate and this would break with one of the
main principles of CSL \fxwarning{citation needed}.
% Limitation: no recursion and why?

\paragraph{Case 2: List manipulation}
As we have banned recursion, we still have no way of iterating through
lists and manipulating them, thus some other language primitives are
needed. Our solution to this problem is simple: we provide three
functions known from functional languages for iterating through and
manipulate lists. In particular, we have implemented \lstinline{foldl},
\lstinline{foldr} and \lstinline{cons} as built-in functions, such to
allow the user to specify a large array of functions expressable with
these primitives.


\chapter{Implementation}
\section{Syntactic extensions}
The syntax of the complete language is defined in the tech report by
Tom Hvitved \cite[p. 13]{hvitved10}. As we are solely interested with
the expression language, we will focus on the relevant extensions.

\begin{align}
  p ::= &\ \ldots || val~x = e || fun~ x (v) = e \\
  e ::= &\ \ldots || \lambda x -> e || e_1\ e_2
\end{align}

\section{Built-in extensions}
As a major motivation, in introducing function expressions through
syntax, was to eliminate the previously built-in list primitives,
\lstinline{max} and \lstinline{min}, we also introduced an explicit
requirement of being able to specify exactly those functions throug
other means. Thus we had to implement a small array of super-primitives,
with which we through the extended expression syntax could specify the
mentioned primitives. Implementing \lstinline{fold}'s as the
super-primitives would not only ensure that we are able to express 
\lstinline{max} and \lstinline{min}, but also be able to introduce more
advanced functions, such as \lstinline{map}, \lstinline{length} and so 
forth; in addition also ensure that the need for recursive operations
in expressions are kept at an absolute minimum.


\section{Type Inference}


\defbibheading{bibliography}{\chapter{Bibliography}}
\printbibliography

\appendix

\chapter{Motivating cases}

\paragraph{Example 1 - Syntactic sugar for recursive contracts.}
Story: Defining recursive contracts can be complicated to define, take
up much space and are hard to read. We now propose syntactic sugar for
defining recursive functions as iterative contracts using
repeat-until/collect-until syntax:

\begin{lstlisting}
repeat <a> A(...)
    within e_{1} after e_{2}
as d until P(d)
\end{lstlisting}
alternatively
\begin{lstlisting}
collect <a> A(...)
    within e_{1} after e_{2}
in d until P(d)

for d being a list of some accumulated elements
     P() being a predicate or list operation.
\end{lstlisting}     
Since the semantics dictate collecting of actions, where a condition is met, in
‘d’, until predicate P(d) == T, the collect-until gives a better intuitive
understanding of it’s functionality.
The repeat-until/collect-until is equivalent to the following construction:
\begin{lstlisting}[escapechar=\#]
repeat(accumulator: #$\tau$#, x: #$\tau’$#, y: #$\tau’$#)
    <a> A(...) @ t
        within x after y
    then     if P(A({...} : accumulator)
        then empty
        else rep(A({...}: accumulator, x+y-t, 0)
\end{lstlisting}

\paragraph{Example 2 - syntactic sugar for parameter-transferring “where”-clauses in contracts.}
Story: Existing “where”-clauses are bloated and verbose. We propose syntactic sugar to alleviate this problem and make the clause simpler and more concise:

\begin{lstlisting}
<seller> A(resource r == goods, receiver b == buyer)
\end{lstlisting}
is interpreted as the old construction:
\begin{lstlisting}
    <seller> A(resource r, receiver b)
        where     r == goods andalso
            b == buyer
\end{lstlisting}

Note also, that the where clause will still be possible to use in the ‘syntactic
sugar’-version, where comparisons are more complicated, such as:

\begin{lstlisting}
<seller> A(resource r == goods, receiver b == buyer, amount a)
        where a = 1.1 * (amount / 2)
\end{lstlisting}

for A being some action, e.g. transferAndDeliver.


\paragraph{Example 3 - Extended expression language and list operators.}
Story: Consider a scenario where a customer B wants her goods to be delivered in
portions of all fragile wares separated from heavy-duty wares, but in the same
contract. This is not immediately possible in the existing system, and can in
cases of more complex separations of all the goods lead to very complex and
hard-to-read contracts. We now propose predicates with list operators following
the template:

\begin{lstlisting}
<seller> Deliver(receiver b == customer, resource r)
    where r == q(goods) p

for q being a list operator, e.g. 'all'.
     p being a list predicate, e.g. \x.x == fragile (lambda x where x == fragile)
\end{lstlisting}

\paragraph{Example 4 - date arithmetic.}
Story: Contract templates allow only relative dates for fields such as
deadlines. An instantiation of a contract will not always be worthwhile, when
deadlines are set to be specific dates, as all arithmetic of constructing
relative times from an absolute date is time consuming and not handled by the
system. We introduce simple date arithmetic to allow the system to perform
run-time conversion of absolute dates to relative time measures.

\begin{lstlisting}
Contract sale : Sale =
    SaleMult(deliveryDate - startDate, ...)
\end{lstlisting}

During interpretation of this expression, deliverDate - startDate, is evaluated
to a relative measure for the field deliveryDeadline, to comply with the
contract language constraints regarding handling of dates.


\subsection{Case 1}
Story: Concrete contract case using list predicates, added syntactic sugar for
“where”-clauses and contract templates using date arithmetic and list operators.
\begin{itemize}
\item Seller agrees to transfer and deliver to buyer. The goods: 1 bed (king
size), 1 chair, 1 box of wine glass.
\item The bed and chair, being heavy duty, must be delivered on or before
2010-08-27. The box of wine glass, being fragile, must be delivered before
2010-08-30 but after 2010-08-27. All goods are summing to the price of 2000 USD.
\item All goods must be paid after first delivery.
\end{itemize}
Before we present the implementations of the system, we will provide the contract and contract instantiation template for the above contract definition.

\end{document}
