\documentclass[10pt,a4paper,final,oneside,openany,article]{memoir}
\usepackage[utf8]{inputenc}
\usepackage[danish, british]{babel}
\usepackage{graphicx}
\usepackage{url}
\usepackage{listings}
\usepackage[draft]{fixme}
\usepackage{amsmath}
%\usepackage[ntheorem, overload]{empheq}
\usepackage{semantic}
\usepackage{color}

% makes sure we can make relative links.
\usepackage[colorlinks]{hyperref}
\usepackage{memhfixc} % hyperlink breaks memoir. This is the fix
\makeatletter
\renewcommand\href[2]{\hyper@linkurl{#2}{#1}}
\makeatother


% Fonts configuration
%  - Palatino and Bitstream Vera Sans Mono for verbatim
\usepackage[T1]{fontenc}
\usepackage{palatino}
\usepackage[sc]{mathpazo} % Math font for Palatino
\usepackage{bera}
\linespread{1.05} % Palatino needs more leading (space between lines)
\usepackage{microtype} % ++

% Commands
\newcommand{\twofig}[6]{
\begin{figure}[htp!]
\begin{center}
\fbox{    \subfigure[#1]{    \includegraphics[width=0.4\textwidth]{#2}}
    \subfigure[#3]{    \includegraphics[width=0.4\textwidth]{#4}    } }
\caption{#5}
\label{#6}
\end{center}
\end{figure}
}

\newcommand{\onefig}[4]{
\begin{figure}[htp!]
\begin{center}
\fbox{    \includegraphics[width= #4 \textwidth]{#1}}
\caption{#2}
\label{#3}
\end{center}
\end{figure}
}

\mathlig{||}{\ |\ }

\newcommand{\kw}[1]{\text{\lstinline{#1}}}
\newcommand*\widefbox[1]{\fbox{\hspace{1em}#1\hspace{1em}}}

% Figure caption setup
\usepackage{caption}
\captionsetup{margin=0pt, font=small, labelfont=bf, format=hang}
\setlength{\abovecaptionskip}{0pt}
%\setlength{\belowcaptionskip}{0pt}

% Bibliography
\usepackage[style=alphabetic,natbib=true]{biblatex}
\bibliography{../bibliography}

\definecolor{kugrey}{rgb}{.4,.4,.4}

% Language syntax highlighting
\lstdefinelanguage{CSL}{ keywords={contract, true, false, if, then,
    else, where, due, immediately, within, after, or, and, andalso,
    empty, fun, val},
  keywordstyle = \textbf,
  commentstyle = \color{kugrey}\textit,
  stringstyle = \ttfamily,
  basicstyle   = \small\ttfamily,
  numberstyle=\tiny,
  sensitive = true,
  morecomment=[s]{/*}{*/},
  morecomment = [l]{//}
}
\lstset{language = CSL}

% Document meta
\title{Extending the POETS \\ Contract Specification Language\\
       \vspace{0.1cm}
        \small{A Summer of POETS Coding Extravaganza}}
\author{Philip Lykke Carlsen (plcplc@gmail.com) \and
        Martin Dybdal (dybber@dybber.dk) \and
        Brian Søborg Mathiasen (soborg@diku.dk)}
\date{27th August 2010}

% Layouting
\setcounter{tocdepth}{3}
\pagestyle{plain}
\setcounter{secnumdepth}{1}

\begin{document}
% \listoffixmes
% \fxnote{Kalder vi det expression language eller predicate language?}

% \newpage

\maketitle

\chapter{Formalities}
This report was written as part of the DIKU course ``POETS Summer of
Code'' and presents a set of extensions to the contract specification
language of the POETS\footnote{POETS abbreviates Process-Oriented
Event-driven Transaction Systems}
\cite{DBLP:journals/jlp/HengleinLSS09} platform. The project was
supervised by Tom Hvitved and our code is available in the default
branch of the POETS Mercurial repository. Together with this report we
have handed in Haddock documentation for POETS system.

\chapter{Introduction}

Recently, Tom Hvitved presented a language for specifying multi-party
contracts \cite{hvitved10} as part of the 3gERP and POETS projects
\cite{3gerp}. The \textit{Contract Specification Language}
(\textit{CSL}) makes it possible to write down sets of commitments
between several parties (e.g. individuals and companies) and combining
those commitments into legally binding agreements, \textit{contracts}.
The language overcomes a number of limitations in earlier published
contract formalisations, for example blame assignment and support for
both relative and absolute temporal constraints.

The language is structured around \textit{clauses} which are combined
into \textit{contract templates} by a set of provided
combinators. These templates can then be instantiated in the running
system. For example a company might describe a sale of goods as a
contract template and when a specific item is sold, this contract
template is instantiated with the information about the article and
its price. At instantiation time all relative temporal constraints are
instantiated to absolute constraints, running from the time of
instantiation. Clauses describe the \textit{actions} or
\textit{transactions} that certain \textit{agents} (contract parties,
e.g. a company) should perform and specifies temporal constraints for
these actions (e.g., a package must be dispatched within 10 days), at
last a clause also contains a predicate detailing which constraints
that should be satisfied (e.g. a customer should transfer the agreed
amount of money). This predicate language (which we will often refer
to as ``the CSL expression language'') was very limited in Tom
Hvitved's incarnation. For instance, while CSL supports lists, the
predicate language had no primitives for manipulating them and
provides no means of defining important abstractions, like functions.

We have extended the predicate language of CSL, on the basis of a
number of contract schemes of practical use, that were impossible to
formulate with the previous version of the language. The extensions
ensure the integrity of the core features of the language described in
\cite{hvitved10}.  In Section \ref{extensions} we will go through the
limitations of the current language that we want to solve. In Section
we will determine what the syntax and semantics of these extensions
should be. Section \ref{implementation} describes how we have
implemented the extensions. We round off our report by explaining how
the extensions were tested (Section \ref{tests}) and some concluding
remarks, including a set of further language features that could be
implemented.

\chapter{Extensions}
\label{extensions} In this section we will go through our extensions
of CSL and at the same time motivating these extensions by describing
contracts that could occur in real world applications.

\paragraph{Case 1: Manipulating lists} In the existing language it is
impossible to modify the contents of lists in any way. The only
available operations on lists were computing its length and comparison
operations. The language thus need more general operators for
performing list manipulations.

An example of a contract that con not be written in the previous
language incarnation, is a sales contract where the order is delivered
in several parcels. When the first parcel is received, the contract
should be updated to reflect that seller only needs to deliver the
remaining items of the order.  We show how this could be implemented
in Appendix \ref{chap:sale_partial_delivery}.

\paragraph{Case 2: Abstractions and polymorphism} The current language
does not provide any way of writing functions on the expression
level. Contracts and clause templates provides abstraction on the
syntactical hierarchy above. Clause templates could be used for for
some amount of code-reuse and abstractions, but it would not be
worthwhile. Functions are important abstractions in all modern
languages and without them code maintenance can become complicated. It
is common knowledge that code duplication should be avoided.
Furthermore, functions should be as general as possible. It should not
be necessary to restrict a length function for lists to only work on
lists of e.g. integers. We thus want ML-style parametric polymorphism
in our language.

Our example in Appendix \ref{chap:sale_partial_delivery} also
illustrates the usefulness of functions for code reuse.

\paragraph{Case 3: Conditional expression.} Decisions can not be taken
on the expression level, only decisions that selects between several
clauses can be taken. Clause-level conditionals could be used for
implementing certain conditions, but for many applications they would
add an amount of repetition that would not be needed with expression
level conditionals.

As a very simple example of the usefulness of expression-level
conditionals, it should be noted that the minimum function could not
be expressed in the original language and was previously a hardcoded
primitive in the expression evaluator.

\chapter{Selecting language primitives}
\label{primitives}
In this section we analyse which language primitives we need to
overcome the problems described in the cases mentioned in
Section \ref{extensions} above. It is desirable to restrict the amount of
language built-in primitives to a minimum, to limit implementation
complexity.

\paragraph{Case 1: List manipulation}
A need for list manipulations implies a need of iteration or recursion
primitives. We could implement recursive functions along with head,
tail and cons as a very minimajl set of primitives, but we must not
allow recursion, as this will make it possible to write potentially
non-terminating functions, and all expressions must terminate. Instead
we need a way of iterating through lists while still guaranteeing that
all expressions will terminate. Our solution to this problem is simple:
we provide three functions known from functional programming languages
for iterating through and manipulate lists. In particular, we have
implemented \lstinline{foldl}, \lstinline{foldr} and \lstinline{cons} as
built-in functions, which allows the user to write the large array of
functions expressible with these primitives.

\paragraph{Case 2: Functions} As CSL is a declarative language, our
functions should be inspired by declarative principles. We therefore
draw from the world of functional programming and define functions as
lambda expressions that acts as primitive values which can be passed
between other functions (higher-order functions). An important
limitation for expressing functions on the expression level is that we
do now allow recursive functions. The important rationale for this is
that contracts containing potentially diverging predicates are
essentially meaningless and therefore void of any legal value. So in
order to assure that any contract that is accepted by the CSL system
is indeed well defined we are restricted from allowing recursion or
any other language construction that entails turing completeness.

\paragraph{Case 3: Conditional expressions}
Expression level conditionals are common features of all functional
languages and thus we simply adopt the notation and semantics from
such languages.

\newpage
\chapter{Implementation}
\label{implementation}
In this section, we will go through the implementation of the extensions
giving an overview of only the most important details of the extensions.
We will explain the syntax for all the extensions, describe built-in
functions, type inferer and evaluator accompanied with the typing and
big-step semantics for the extended language. Finally, we will highlight
a number of test cases documenting the expressions function as expected.

\section{Syntactic extensions}
\begin{figure}
  \begin{align*}%[box=\widefbox]
    p ::= &~ \ldots || \kw{val}~ id : type = e \tag{CSL terms} \\
       & \quad\quad || \kw{fun}~ id : type~ (id_1 : type_1, id_2 : type_2, \ldots, id_n : type_n) = e  \\
    e ::= &~ \ldots || \lambda id_1~ id_2~ \ldots~ id_n -> e || e_1\ e_2  \tag{CSL predicates}\\
       & \quad\quad || \kw{if}~ c~ \kw{then}~ e_1~ \kw{else}~ e_2 \\ 
    type ::= &~ \ldots || type_1 -> type_2 ||
    typevar \tag{CSL types}
  \end{align*}
  \caption{Our extensions to the CSL grammar}
\label{fig:bnf}
\end{figure}

The grammar of the complete language is defined in the tech report by
Tom Hvitved \cite[page 13, Figure 1]{hvitved10}, our extensions can be seen in
Figure \ref{fig:bnf}.

We have extended the set of CSL terms, $p$, to include \textit{value}
and \textit{function definitions}.  As can be seen, our only syntactic
extensions to the CSL expression language (nonterminal $e$) are lambda
expressions, function application and conditional
expressions\footnote{Actually, negation is also our addition. It
  existed only by coincidence in the previous version.} %negationaddition
The multi
parametered expressions unfold to several embedded lambda expressions
and a function definition unfolds to a value binding of a lambda
expression.

Type expressions are extended with function types and type
variables to accomodate for the newly introduced lambda abstractions.
We distinguish type variables from ordinary types by
insisting that all ordinary types start with a majuscule and all type
variables start with a minuscule.

These extensions are made to the
\href{doc/html/poets/Poets-Contracts-Language-CSL-Parser.html}{Poets.Contracts.Language.CSL.Parser} module of Poets. The most
interesting extensions are function application which are implemented
as a left-associative operator and \ldots.


\section{Built-in functions}
A small set of built-in functions are necessary to define all the
functions that we want to be expressible. As we discussed in Section
\ref{primitives}, the list operations \lstinline{foldl},
\lstinline{foldr} and \lstinline{cons} are needed for writing other
list operators like map or filter. In addition the deadline handling
code of the CSL engine also needs the functions: \lstinline{min},
\lstinline{max} and \lstinline{ceil}. We can define \lstinline{min}
and \lstinline{max} using our \lstinline{if-then-else} statement, but
ceiling needs to be included in our list of built-in functions.

We have implemented these four mentioned \textit{super primitives}
(\lstinline{foldl}, \lstinline{foldr}, \lstinline{cons} and
\lstinline{ceil}) in the evaluator algebra by writing down a their
syntax trees directly and their generated type constraints are also
added directly to the type inference algebra.

\section{Type Inference}
\begin{figure}
  \begin{equation*}
    \frac{
      \Gamma |- c_1 : ~\kw{Bool}
      \quad \Gamma |- e_1 : \tau_1
      \quad \Gamma |- e_1 : \tau_2
    }{
      \Gamma |- \kw{if}~ c_1 ~\kw{then}~ e_1 ~\kw{else}~ e_2 : \tau_1
    }(\tau_1=\tau_2)\label{eq:type_cond}
  \end{equation*}

\begin{equation*}
  \frac{
    \Gamma\{id : \tau_1\} |- e : \tau_2
  }{
    \Gamma |- \lambda id -> e : \tau_1 -> \tau_2
  }
  \quad \quad
  \frac{
    \Gamma |- e_1 : \tau_1 -> \tau_2
    \quad \Gamma |- e_2 : \tau_1
  }{
    \Gamma |- e_1~ e_2 : \tau_2
  }\label{eq:type_apply}
\end{equation*}

\caption{Typing rules for the extended language}
\label{fig:typing_rules}
\end{figure}
The original CSL implementation already included support for type
variables, parametric polymorphism and also subtyping polymorphism,
but had to be extended to include function types.  This extended is
implemented using method explained in \textit{Data types à la carte},
\cite{swierstra2008data}, which are an integral part of the structure
of the POETS platform. Our extension is implemented by the EType
data type in \href{doc/html/poets/Poets-Contracts-Language-CSL-ALaCarte-EType.html}{Poets.Contracts.Language.CSL.ALaCarte.EType} and the new
CType declaration is then
\begin{lstlisting}
  type CType = Term (TypeConstant :+: TypeList :+: TypeVar :+: EType)
\end{lstlisting}

We have also extended the type inference algebra for CSL expressions
to include the constraints generated by the three typing rules shown
in Figure \ref{fig:typing_rules}. The typing rules of the original
language is found at \cite[page 17, Figure
3]{hvitved10}. \fixme{examplify subtype constraint, for example cond <: TBool}
In addition we have extended the default typing environment $\Gamma$
to include the types of our four built-in functions:
\lstinline{foldl}, \lstinline{foldr}, \lstinline{cons} and
\lstinline{ceil}.

As the current type inference was developed by Tom Hvitved while we
were working on the project, it is still not completely
functional. Specifically, the constraint solver for the CSL type
inferer is yet to be implemented. This means that we have not been
able to test whether our generated constraints are correct and whether
they suffice to perform the inference. We thus can not give any
guarentees on the implemented functionality of this part.

% These rules ensure that it is impossible to make recursive functions
% by writing fix-point combinators.

\section{Type Checking}
We had to extend the type checker to also type check the value
bindings.  This is done by type checking the definitions in sequential
order of their appearence in a file, extending the typing environment
when each definition has been processed. This ensures that it is
impossible to make self-referencing, and thereby recursive,
definitions.  This also includes mutually recursive definitions.

\section{Evaluating expressions}
An important addition to the evaluation algebra in CSL is the use of
environments. We have extended the evaluator monad with an environment
which is passed along where appropriate. The need for an environment is
facilitated by the lambda abstraction; this also means the semantics of
the expression evaluation is extended with this environment, while only
utilized by the lambda evaluation algebra instance. Thus, the semantics
of the extended language are as illustrated in Figure
\ref{fig:bigstep_semantics}. There is no need to list the semantic for
lambda expressions, due to it's axiomatic nature.

\begin{figure}
\begin{equation*}
\frac{
  \sigma |- b \Downarrow \kw{false}
  \quad \sigma |- c_2 \Downarrow v_2
}{
  \sigma |- \kw{if} ~b ~\kw{then} ~c_1 ~\kw{else} ~c_2 \Downarrow v_2
}
\quad \quad
\frac{
  \sigma |- b \Downarrow \kw{true}
  \quad \sigma |- c_1 \Downarrow v_1
}{
  \sigma |- \kw{if} ~b ~\kw{then} ~c_1 ~\kw{else} ~c_2 \Downarrow v_1
}\label{eq:eval_condition}
\end{equation*}

\begin{equation*}
\frac{
  \sigma |- e_1 \Downarrow \lambda x \rightarrow e_3
  \quad \sigma |- e_2 \Downarrow v_1
  \quad \sigma[x\mapsto v_1] |- e_3 \Downarrow v_2
}{
  \sigma |- e_1 ~e_2 \Downarrow v_2
}\label{eq:eval_apply}
\end{equation*}

\caption{Big-step semantics for the extended language}
\label{fig:bigstep_semantics}
\end{figure}

In particular, the class for the evaluation algebra, in
\href{doc/html/poets/Poets-Data-ALaCarte-Algebras-Evaluator.html}{Poets.Data.ALaCarte.Algebras.Evaluator}, is listed as such:
\lstset{language=Haskell}
\begin{lstlisting}
class (Functor f, Error (err ())) => Evaluator f env err v where
    evalAlg :: f (EI env (err ()) (Term v)) -> EI env (err ()) (Term v)
\end{lstlisting}
The most important extension to this class is the environment, such that
each instance of \lstinline{evalAlg} carries an environment as dictacted
by the above semantics. However, with such an extension to the class, we
also need to be able to evaluate using that environment. For this
purpose we introduce the \lstinline{EI} monad on ``both sides'' of the
evaluator class. This in turn, will let us define the \lstinline{EI}
monad and monad execution helper function as such:
\begin{lstlisting}
-- |Monad for evaluation with an environment.
type EI env err = ReaderT env (Either err)

-- |Execute evaluation monad and obtain the result.
runEI :: EI env err v -> env -> Either err v
runEI = runReaderT
\end{lstlisting}

Another important aspect of the evaluator algebra is how we handle
lambda expressions. Knowing that we have to use an environment for this
evaluation, we can make use of the beforementioned constructions to
ensure the integrity of the environment throughout the calculation.
The complete algebra for the evaluation in
\href{doc/html/poets/Poets-Contracts-Language-CSL-ALaCarte-Exp-Algebras-Evaluator.html}{Poets.Contracts.Language.CSL.ALaCarte.Exp.Algebras.Evaluator} of lamba
expressions is:
\begin{lstlisting}
    evalAlg ELambda{expLArg = var,
                    expLBody = e} = do
      env <- ask
      return . inject $
                 ((VLambda $ \v ->
                       let env' = Map.insert var v (prj env)
                       in case runEI e (upd env' env) of
                            Left err -> evalLambdaError err
                            Right v -> v) :: EValue (Term v) (Term v))
\end{lstlisting}
We inject a VLambda construction in which we bind the argument
\lstinline{var} in the environment \lstinline{env'}, run the body of the
lambda using the updated environment, yeilding a \lstinline{Left} or
\lstinline{Right} accordingly. In essence, \lstinline{VLambda} is an
\href{doc/html/poets/Poets-Contracts-Language-CSL-ALaCarte-EValue.html}{Poets.Contracts.Language.CSL.ALaCarte.EValue} \lstinline{EValue} function
type \lstinline{f -> e} in :
\begin{lstlisting}
data EValue f e = VLambda (f -> e)
\end{lstlisting}

%% describe the purpose of VLambda.
in \href{doc/html/poets/Poets-Contracts-Language-CSL-AST.html}{Poets.Contracts.Language.CSL.AST}
\begin{lstlisting}
newtype CValue = CValue {unCValue :: (Term (Value :+: EValue CValue))}
\end{lstlisting}


\lstset{language=CSL} % reset language to CSL.
% Extensions of the evaluator monad with environments
% Implementation of VLambda and extension of CValues

\section{Testing}
\label{tests}
A specifically stated requirement from the 3gERP-team was that every
project team must perform unit tests of the extensions that they
produce. To meet this end we have created unit tests of the parser
module and the evaluator module using the \emph{HUnit} framework.

The tests of the parser each compare a piece of csl syntax with a
directly constructed AST\footnote{Abstract Syntax Tree}.  The tests
deal with the parsing of

\begin{itemize}
  \item value bindings
  \item function application
  \item lambda definitions
  \item type annotations
\end{itemize}

The tests we have performed are only constructive, in the sense that
only valid pieces of syntax are tested to parse to their expected
ASTs.

\chapter{Conclusion}
We have extended the Contract Specification Language of the POETS
platform to include primitives for writing lambda expressions,
conditional expressions, as well as defining named values which includes
functions as well as built-in list manipulation primitives. Given these
extensions, we have established a solid foundation for providing rich
functionality through the expression language, which will give more
flexibility in contract specifications.
Finally, the core features of the original language \cite[page
4]{hvitved10} have been preserved in the extended language.

%While a large portion of the extensions have been tested and verified
%the type checking is still of unknown stability, as the constraint
%solver for the CSL type inferer is not yet implemented at the time of
%writing.

\section{Future work}
As we have worked with the language we have had additional ideas to
features that could be \textit{nice to have} in the contract language.
Note that these suggestions are stricly related with the CSL expression
language, for suggestions of the greater scheme of CSL, refer to
\cite{hvitved10}.
\begin{itemize}
%\item \textbf{A module system}. 
%\item \textbf{\lstinline{let}-bindings in expressions}. (syntactic sugar for
%  lambda binding variable).
\item \textbf{Additional datatypes}. For example pairs or dynamic
  creation of new record types. To provide a richer library of primitive
  datatypes.
\item \textbf{Repeat-loop to repetitive clauses} (e.g. an action that
  has to be taken every 14 days). The purpose would be to give a better
  intuitive understanding of recursive contracts. As such, this
  extension can be implemented as purely syntactic sugar.
\item \textbf{Elimination of parameter-passing \lstinline{where}-clauses.} It is
  observed that a large amount of redundancy is performed in
  \lstinline{where}-clauses. Such an extension will be able to eliminate
  verbose operations as seen in the existing contract clauses:
  \begin{lstlisting}
  ...
  <agent> SomeAction(arguments)
  where g == goods andalso
        a == amount andalso
        c == currency
        ...
  \end{lstlisting}
  This extension will eliminate a large amount of this redundancy as
  the expressions can be collapsed into the arguments of the actions.
\end{itemize}

\defbibheading{bibliography}{\chapter{Bibliography}}
\printbibliography


\newpage
    \appendix
\chapter{Contract for partial deliveries: sale\_partial\_delivery.csl}
\label{chap:sale_partial_delivery}
\lstinputlisting{sale_partial_delivery.csl}

\end{document}
