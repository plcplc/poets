\documentclass[10pt,a4paper,final,oneside,openany,article]{memoir}
\usepackage[utf8]{inputenc}
\usepackage[danish, british]{babel}
\usepackage{graphicx}
\usepackage{url}
\usepackage{listings}
\usepackage[draft]{fixme}
\usepackage{amsmath}
%\usepackage[ntheorem, overload]{empheq}
\usepackage{semantic}
\usepackage{color}

% makes sure we can make relative links.
\usepackage[colorlinks, filecolor=blue, urlcolor=blue, citecolor=blue, linkcolor=black]{hyperref}
\usepackage{memhfixc} % hyperlink breaks memoir. This is the fix
\makeatletter
\renewcommand\href[2]{\hyper@linkurl{#2}{#1}}
\makeatother


% Fonts configuration
%  - Palatino and Bitstream Vera Sans Mono for verbatim
\usepackage[T1]{fontenc}
\usepackage{palatino}
\usepackage[sc]{mathpazo} % Math font for Palatino
\usepackage{bera}
\linespread{1.05} % Palatino needs more leading (space between lines)
\usepackage{microtype} % ++

% Commands
\newcommand{\twofig}[6]{
\begin{figure}[htp!]
\begin{center}
\fbox{    \subfigure[#1]{    \includegraphics[width=0.4\textwidth]{#2}}
    \subfigure[#3]{    \includegraphics[width=0.4\textwidth]{#4}    } }
\caption{#5}
\label{#6}
\end{center}
\end{figure}
}

\newcommand{\onefig}[4]{
\begin{figure}[htp!]
\begin{center}
\fbox{    \includegraphics[width= #4 \textwidth]{#1}}
\caption{#2}
\label{#3}
\end{center}
\end{figure}
}

\mathlig{||}{\ |\ }

\newcommand{\kw}[1]{\text{\lstinline{#1}}}
\newcommand*\widefbox[1]{\fbox{\hspace{1em}#1\hspace{1em}}}

% Figure caption setup
\usepackage{caption}
\captionsetup{margin=0pt, font=small, labelfont=bf, format=hang}
\setlength{\abovecaptionskip}{0pt}
%\setlength{\belowcaptionskip}{0pt}

% Bibliography
\usepackage[style=alphabetic,natbib=true]{biblatex}
\bibliography{../bibliography}

\definecolor{kugrey}{rgb}{.4,.4,.4}

% Language syntax highlighting
\lstdefinelanguage{CSL}{ keywords={contract, true, false, if, then,
    else, where, due, immediately, within, after, or, and, andalso,
    empty, fun, val},
  keywordstyle = \textbf,
  commentstyle = \color{kugrey}\textit,
  stringstyle = \ttfamily,
  basicstyle   = \small\ttfamily,
  numberstyle=\tiny,
  sensitive = true,
  morecomment=[s]{/*}{*/},
  morecomment = [l]{//}
}
\lstset{language = CSL}

% Document meta
\title{Extending the POETS \\ Contract Specification Language\\
       \vspace{0.1cm}
        \small{A Summer of POETS Coding Extravaganza}}
\author{Philip Lykke Carlsen (plcplc@gmail.com) \and
        Martin Dybdal (dybber@dybber.dk) \and
        Brian Søborg Mathiasen (soborg@diku.dk)}
\date{27th August 2010}

% Layouting
\setcounter{tocdepth}{3}
\pagestyle{plain}
\setcounter{secnumdepth}{1}

\begin{document}
% \listoffixmes
% \fxnote{Kalder vi det expression language eller predicate language?}

% \newpage

\maketitle

\chapter{Formalities}
This report was written as part of the DIKU course ``POETS Summer of
Code'' and presents a set of extensions to the contract specification
language of the POETS\footnote{POETS abbreviates Process-Oriented
  Event-driven Transaction Systems}
platform\cite{DBLP:journals/jlp/HengleinLSS09}. The project was
supervised by Tom Hvitved and our code is available in the default
branch of the POETS Mercurial repository. Together with this report we
have handed in Haddock documentation for POETS system.

\chapter{Introduction}

Recently, Tom Hvitved presented a language for specifying multi-party
contracts \cite{hvitved10} as part of the 3gERP and POETS projects
\cite{3gerp, DBLP:journals/jlp/HengleinLSS09}. The \textit{Contract Specification Language}
(\textit{CSL}) makes it possible to write down sets of commitments
between several parties (e.g. individuals and companies) and combining
those commitments into legally binding agreements, \textit{contracts}.
The language overcomes a number of limitations in earlier published
contract formalisations, for example blame assignment and support for
both relative and absolute temporal constraints.

The language is structured around \textit{clauses} which are combined
into \textit{contract templates} by a set of provided
combinators. These templates can then be instantiated in the running
system. For example a company might describe a sale of goods as a
contract template and when a specific item is sold, this contract
template is instantiated with the information about the article and
its price. At instantiation time all relative temporal constraints are
instantiated to absolute constraints, running from the time of
instantiation. Clauses describe the \textit{actions} or
\textit{transactions} that certain \textit{agents} (contract parties,
e.g. a company) should perform and specifies temporal constraints for
these actions (e.g., a package must be dispatched within 10 days). At
last, a clause also contains a predicate detailing which constraints
that should be satisfied, for example that a customer should transfer
an agreed amount of money. This predicate language (which we will
often refer to as ``the CSL expression language'') was very limited in
Tom Hvitved's incarnation. For instance, while CSL supports lists, the
predicate language had no primitives for manipulating them and
provides no means of defining important abstractions, like functions.

We have extended the predicate language of CSL, on the basis of a
number of contract schemes of practical use, that were impossible to
formulate with the previous version of the language. The extensions
ensure the integrity of the core features of the language described in
\cite{hvitved10}. 

In the next section, Section \ref{extensions}, we will go through the
limitations of the current language that we want to solve. In Section
\ref{primitives} we will determine what the syntax and semantics of
these extensions should be. Section \ref{implementation} describes how
we have implemented the extensions. We round off our report by
explaining how the extensions were tested (Section \ref{tests}) and
some concluding remarks, including a set of further language features
that could be implemented.

\chapter{Extensions}
\label{extensions} In this section we will go through our extensions
of CSL and at the same time motivating these extensions by describing
contracts that could occur in real world applications.

\paragraph{Case 1: Manipulating lists} In the existing language it was
impossible to modify the contents of lists in any way. The only
available operations on lists were computing its length and comparison
operations. The language thus need more general operators for
performing list manipulations.

An example of a contract that could not be written in the previous
language incarnation, is a sales contract where the order is delivered
in several parcels. When the first parcel is received, the contract
should be updated to reflect that seller only needs to deliver the
remaining items of the order.  We show how this could be implemented
in Appendix \ref{chap:sale_partial_delivery}.

\paragraph{Case 2: Abstractions and polymorphism} The current language
does not provide any way of writing functions on the expression
level. Contracts and clause templates provides abstraction on the
syntactical hierarchy above. Clause templates could be used for
some amount of code-reuse and abstractions, but it would not be
worthwhile. Functions are important abstractions in all modern
languages and without them code maintenance can become complicated. It
is common knowledge that code duplication should be avoided.
Furthermore, functions should be as general as possible. It should not
be necessary to restrict a length function for lists to only work on
lists of e.g. integers. We thus want ML-style parametric polymorphism
in our language.

Our example in Appendix \ref{chap:sale_partial_delivery} also
illustrates the usefulness of functions for code reuse.

\paragraph{Case 3: Conditional expression.} Decisions can not be taken
on the expression level, only decisions that selects between several
clauses can be taken. Clause-level conditionals could be used for
implementing certain conditions, but for many applications they would
add an amount of repetition that would not be needed with expression
level conditionals.

As a very simple example of the usefulness of expression-level
conditionals, it should be noted that the minimum function could not
be expressed in the original language and was previously a hardcoded
primitive in the expression evaluator.

\chapter{Selecting language primitives}
\label{primitives}
In this section we analyse which language primitives we need to
overcome the problems described in the cases mentioned in
Section \ref{extensions} above. It is desirable to restrict the amount of
language built-in primitives to a minimum, to limit implementation
complexity.

\paragraph{Case 1: List manipulation}
A need for list manipulations implies a need of iteration or recursion
primitives. We could implement recursive functions along with head,
tail and cons as a very minimal set of primitives, but we must not
allow recursion, as this will make it possible to write potentially
non-terminating functions, and all expressions must terminate. Instead
we need a way of iterating through lists while still guaranteeing that
all expressions will terminate. Our solution to this problem is simple:
we provide three functions known from functional programming languages
for iterating through and manipulate lists. In particular, we have
implemented \lstinline{foldl}, \lstinline{foldr} and \lstinline{cons} as
built-in functions, which allows the user to write the large array of
functions expressible with these primitives.

\paragraph{Case 2: Functions} As CSL is a declarative language, our
functions should be inspired by declarative principles. We therefore
draw from the world of functional programming and define functions as
lambda expressions that acts as primitive values which can be passed
between those same functions (higher-order functions). An important
limitation for expressing functions on the expression level is that we
do not allow recursive functions. The important rationale for this is
that contracts containing potentially diverging predicates are
essentially meaningless and therefore void of any legal value. So in
order to assure that any contract that is accepted by the CSL system
is indeed well defined we are restricted from allowing recursion or
any other language construction that entails Turing completeness.

\paragraph{Case 3: Conditional expressions}
Expression level conditionals are common features of all functional
languages and thus we simply adopt the notation and semantics from
such (ML-style) languages.

\newpage
\chapter{Implementation}
\label{implementation}
In this section, we will go through the implementation of the extensions
giving an overview of only the most important details of the extensions.
We will explain the syntax for all the extensions, describe built-in
functions, type inferer and evaluator accompanied with the typing and
big-step semantics for the extended language. Finally, we will highlight
a number of test cases documenting the expressions function as expected.

\section{Syntactic extensions}
\begin{figure}
  \begin{align*}%[box=\widefbox]
    p ::= &~ \ldots || \kw{val}~ id : type = e \tag{CSL statements} \\
       & \quad\quad || \kw{fun}~ id : type~ (id_1 : type_1, id_2 : type_2, \ldots, id_n : type_n) = e  \\
    e ::= &~ \ldots || \lambda id_1~ id_2~ \ldots~ id_n -> e || e_1\ e_2  \tag{CSL predicates}\\
       & \quad\quad || \kw{if}~ c~ \kw{then}~ e_1~ \kw{else}~ e_2 \\ 
    type ::= &~ \ldots || type_1 -> type_2 ||
    typevar \tag{CSL types}
  \end{align*}
  \caption{Our extensions to the CSL grammar}
\label{fig:bnf}
\end{figure}

The grammar of the complete language is defined in the tech report by
Tom Hvitved \cite[page 13, Figure 1]{hvitved10}, our extensions can be seen in
Figure \ref{fig:bnf}.

We have extended the set of CSL statements, $p$, to include \textit{value}
and \textit{function definitions}.  As can be seen, our only syntactic
extensions to the CSL expression language (nonterminal $e$) are lambda
expressions, function application and conditional
expressions\footnote{Actually, negation is also our addition. It
  existed only by coincidence in the previous version.} %negationaddition
The multi
parametered expressions unfold to several embedded lambda expressions
and a function definition unfolds to a value binding of a lambda
expression.

Type expressions are extended with function types and type
variables to accomodate for the newly introduced lambda abstractions.
We distinguish type variables from ordinary types by
insisting that all ordinary types start with a majuscule and all type
variables start with a minuscule.

These extensions are made to the
\href{doc/html/poets/Poets-Contracts-Language-CSL-Parser.html}{Poets.Contracts.Language.CSL.Parser}
module of Poets. A possibly interesting extension (implementation
wise) are function application which are implemented as a
left-associative $\epsilon$-operator.

\section{Built-in functions}
A small set of built-in functions are necessary to define all the
functions that we want expressible. As we discussed in Section
\ref{primitives}, the list operations \lstinline{foldl},
\lstinline{foldr} and \lstinline{cons} are needed for writing other
list operators like \lstinline{map} or \lstinline{filter}. In addition the deadline handling
code of the CSL engine also needs the functions: \lstinline{min},
\lstinline{max} and \lstinline{ceil}. We can define \lstinline{min}
and \lstinline{max} using our \lstinline{if-then-else} construct, but
ceiling needs to be included in our list of built-in functions.

We have implemented these four mentioned \textit{super primitives}
(\lstinline{foldl}, \lstinline{foldr}, \lstinline{cons} and
\lstinline{ceil}) in the evaluator algebra by writing down a their
syntax trees directly and their generated type constraints are also
added directly to the type inference algebra.

\section{Type Inference}
\begin{figure}
  \begin{equation*}
    \frac{
      \Gamma |- e_1 : ~\alpha => C_1
      \quad \Gamma |- e_2 : \beta => C_2
      \quad \Gamma |- e_3 : \gamma => C_3
    }{
      \Gamma |- ~\kw{if}~ e_1 ~\kw{then}~ e_2 ~\kw{else}~ e_3 : \gamma => C[Bool/\alpha][\gamma/\beta] 
    }(C = C_1 \cup C_2 \cup C_3 \text{ and with fresh vars } \alpha, \gamma)
\label{eq:type_cond}
  \end{equation*}

\begin{equation*}
  \frac{
    \Gamma\{id : \alpha\} |- e : \beta => C
  }{
    \Gamma |-  \lambda id -> e : \alpha -> \beta => C
  }
\label{eq:type_lambda}
  \end{equation*}

\begin{equation*}
  \frac{
    \Gamma |- e_1 : \alpha => C_1
    \quad \Gamma |- e_2 : \beta => C_2
  }{
    \Gamma |- e_1~ e_2 : \gamma => C[(\beta -> \gamma)/\alpha]
  }(C = C_1 \cup C_2 \text{ and with fresh var } \gamma)
\label{eq:type_apply}
\end{equation*}

\caption{Typing rules for the extended language, showing how type
  constraints are generated using the judgment $\Gamma |- e : \alpha
  => C$}
\label{fig:typing_rules}
\end{figure}
The original CSL implementation already included support for type
variables, parametric polymorphism and also subtyping polymorphism,
but had to be extended to include function types.  This extension is
implemented using the method explained in \textit{Data types à la
  carte}, \cite{swierstra2008data}, which are an integral part of the
architecture of the POETS platform. Our extension is implemented by
the EType data type in
\href{doc/html/poets/Poets-Contracts-Language-CSL-ALaCarte-EType.html}{Poets.Contracts.Language.CSL.ALaCarte.EType}
and the new CType declaration is then:
\begin{lstlisting}[language=Haskell]
  -- |Types in CSL are regular POETS types extended with type variables
  -- and function types.
  type CType = Term (TypeConstant :+: TypeList :+: TypeVar :+: EType)
\end{lstlisting}

We have also extended the type inference algebra for CSL expressions
to include the constraints generated by the three typing rules shown
in Figure \ref{fig:typing_rules}. Our judgement $\Gamma |- e : \alpha
=> C$ should be read as ``given typing environment $\Gamma$,
expression $e$ will get type $\alpha$ and generate constraints
$C$. The typing rules of the original language is found at \cite[page
17, Figure 3]{hvitved10}, though not mentioning which contraints are
generated.  In addition we have extended the default typing
environment $\Gamma$ to include the types of our four built-in
functions: \lstinline{foldl}, \lstinline{foldr}, \lstinline{cons} and
\lstinline{ceil}.  These extensions are available in
\href{doc/html/poets/Poets-Contracts-Language-CSL-ALaCarte-Exp-Algebras-TypeInferer.html}{Poets.Contracts.Language.CSL.ALaCarte.Exp.Algebras.TypeInferer}

As the current type inference was developed by Tom Hvitved while we
were working on the project, it is still not completely
functional. Specifically, the constraint solver for the CSL type
inferer is yet to be implemented. This means that we have not been
able to test whether our generated constraints are correct and whether
they suffice to perform the inference. We thus can not give any
guarentees on the implemented functionality of this part.

% These rules ensure that it is impossible to make recursive functions
% by writing fix-point combinators.

\section{Type Checking}
We had to extend the type checker to also type check the value
bindings.  This is done by type checking the definitions in sequential
order of their appearence in a file, extending the typing environment
when each definition has been processed. This ensures that it is
impossible to make self-referencing, and thereby recursive,
definitions.  This also includes mutually recursive definitions.
This extension was done in the module
\href{doc/html/poets/Poets-Contracts-Language-CSL-TypeChecker.html}{Poets.Contracts.Language.CSL.TypeChecker}.

\section{Evaluating expressions}

\begin{figure}
\begin{equation*}
\frac{
  \sigma |- b \Downarrow \kw{false}
  \quad \sigma |- c_2 \Downarrow v_2
}{
  \sigma |- \kw{if} ~b ~\kw{then} ~c_1 ~\kw{else} ~c_2 \Downarrow v_2
}
\quad \quad
\frac{
  \sigma |- b \Downarrow \kw{true}
  \quad \sigma |- c_1 \Downarrow v_1
}{
  \sigma |- \kw{if} ~b ~\kw{then} ~c_1 ~\kw{else} ~c_2 \Downarrow v_1
}\label{eq:eval_condition}
\end{equation*}

\begin{equation*}
\frac{
  \sigma |- e_1 \Downarrow \lambda x \rightarrow e_3
  \quad \sigma |- e_2 \Downarrow v_1
  \quad \sigma[x\mapsto v_1] |- e_3 \Downarrow v_2
}{
  \sigma |- e_1 ~e_2 \Downarrow v_2
}\label{eq:eval_apply}
\end{equation*}

\caption{Big-step semantics for the extended language}
\label{fig:bigstep_semantics}
\end{figure}
To implement our extensions to the language, we need to add a new kind
of values to set of possible results of the evaluator: the lambda
abstraction. This extension of values is implemented by the new value
type \lstinline{EValue} (extended values), with the type constructor
\lstinline{VLambda} for lambda abstractions. This is found in the
module \href{doc/html/poets/Poets-Contracts-Language-CSL-ALaCarte-EValue.html}{Poets.Contracts.Language.CSL.ALaCarte.EValue} and implemented
as:
\begin{lstlisting}[language=Haskell]
data EValue f e = VLambda (f -> e)
\end{lstlisting}


It should be noticed that we make a difference between the argument
type and the result type of the function implementing the
abstraction. This was necessary to be able to write a
\lstinline{Functor} instance for the data type, which is a requirement
if it should be used in a \textit{Data types à la carte}
setting. Using this, we can extend the POETS values to include lambda
abstractions, by redefining \lstinline{CValue} as
(done in \href{doc/html/poets/Poets-Contracts-Language-CSL-AST.html}{Poets.Contracts.Language.CSL.AST}):
\begin{lstlisting}[language=Haskell]
newtype CValue = CValue {unCValue :: (Term (Value :+: EValue CValue))}
\end{lstlisting}

To implement $\beta$-reduction of lambda abstractions we need a way to
bind values in the body of the abstraction. This can be done in two
ways. Either we can directly substitute our argument directly into the
body, $\alpha$-converting variables where appropriate. Alternative, we
can use a binding environment which we pass to the evaluator function,
possibly extending it before evaluating subterms.

We have selected to use the environment based approach, as we find
this most modular. These environment based semantics of the extended
language are illustrated in Figure \ref{fig:bigstep_semantics}. As
lambda abstractions are essentially values, there is no need to list
their semantic. It should be noted that the evaluator implementation
does not completely follow the big-step semantics of application. We
will now go through some implementation details and note how our
implementation differs from the semantics above.

In particular, the evaluator monad for the evaluation algebra of
\href{doc/html/poets/Poets-Data-ALaCarte-Algebras-Evaluator.html}{module Poets.Data.ALaCarte.Algebras.Evaluator} is extended to carry
with it an extendable environment. This is done using the
\lstinline{ReaderT} monad transformer:
\begin{lstlisting}[language=Haskell]
-- |Monad for evaluation with an environment.
type EI env err = ReaderT env (Either err)

-- |Execute evaluation monad and obtain the result.
runEI :: EI env err v -> env -> Either err v
runEI = runReaderT
\end{lstlisting}


This also dictates a modification of how evaluation proceeds with
in the evaluator algebra:
\begin{lstlisting}[language=Haskell]
class (Functor f, Error (err ())) => Evaluator f env err v where
    evalAlg :: f (EI env (err ()) (Term v)) -> EI env (err ()) (Term v)
\end{lstlisting}
The noteworthy modification of this type class is the occurence of the
environment in the argument type to \lstinline{evalAlg}. This is
necessary as algebra evaluation is essentially performed bottom up
(when following the \textit{Data types à la carte} definition of
\lstinline{foldTerm}). Thus to be able to bind variables in these
subterms afterwards, the returned value should not be the result of
the computation, but the monadic computation that given an environment
calculates the term.

\begin{figure}
  \begin{lstlisting}[language=Haskell]
    evalAlg ELambda{expLArg = var,
                    expLBody = e} = do
      env <- ask
      return . inject $
                 ((VLambda $ \v ->
                       let env' = Map.insert var v (prj env)
                       in case runEI e (upd env' env) of
                            Left err -> evalLambdaError err
                            Right v -> v) :: EValue (Term v) (Term v))
  \end{lstlisting}
  \caption{Evaluation algebra for lambda abstractions}
  \label{fig:eval_lambda}
\end{figure}
Now, to implement lambda expressions in this environment based
algebra, we have to return a new lambda expression where the body is
executed in a modified version of the environment where the lambda
expression occurs, essentially constructing a closure. Our
implementation of this strategy is shown in \ref{fig:eval_lambda}.

The complete algebra for expression evaluation is found in the module
\href{doc/html/poets/Poets-Contracts-Language-CSL-ALaCarte-Exp-Algebras-Evaluator.html}{Poets.Contracts.Language.CSL.ALaCarte.Exp.Algebras.Evaluator}.

\chapter{Testing}
\label{tests}
A specifically stated requirement from the 3gERP-team was that every
project team must perform unit tests of the extensions that they
produce. To meet this end we have created unit tests of the parser
module and the evaluator module using the \emph{HUnit} framework. As
previously noted, we have been unable to test the type inferer, as the
constraint solver is yet to be implemented.

The tests of the parser each compare a piece of CSL syntax with a
directly constructed syntax tree.  The tests deal with the parsing of

\begin{itemize}
  \item Value bindings,
  \item Function application,
  \item Lambda expressions, and
  \item Type annotations
\end{itemize}

The tests we have performed are only constructive, in the sense that
only valid pieces of syntax are tested to parse to their expected
ASTs.

The tests of the evaluator performs a similar type of tests to assure
that the added expressions evaluate to the expected values.

\section{Listing of modifications and additions}
Note: All modifications have been carried out in the haskell-directory
of the Mercurial repository

\vspace{2em}
\begin{verbatim}
examples/contracts/function-library.csl
examples/contracts/installmentSale.csl
examples/contracts/lease.csl
examples/contracts/master.csl
\end{verbatim}

The example contracts have been updated to use the extended syntax,
and along with them is now a small library of common functions,
defined using the new primitives.

\vspace{2em}
\begin{verbatim}
poets.cabal
\end{verbatim}

The Cabal-file has been suitably expanded to include new files.


%TODO: /src/Poets/Contracts.hs
%TODO: /src/Poets/Contracts/Language/CSL.hs

%\href{doc/html/poets/Poets-Contracts-Language-CSL-AST.html}{Poets.Contracts.Language.CSL.AST}
\vspace{2em}
\noindent\href{doc/html/poets/Poets-Contracts-Language-CSL-ALaCarte-EType.html}{src/Poets/Contracts/Language/CSL/ALaCarte/EType.hs} \\
\href{doc/html/poets/Poets-Contracts-Language-CSL-ALaCarte-EValue.html}{src/Poets/Contracts/Language/CSL/ALaCarte/EValue.hs} \\
\href{doc/html/poets/Poets-Contracts-Language-CSL-ALaCarte-Exp.html}{src/Poets/Contracts/Language/CSL/ALaCarte/Exp.hs} \\
\href{doc/html/poets/Poets-Contracts-Language-CSL-AST.html}{src/Poets/Contracts/Language/CSL/AST.hs} \\

These files contain the extensions to the AST at the expression level,
except for \verb+AST.sh+, which extends the AST on the top level of
the syntax.  They include the AST definitions of lambda expressions,
if-then-else conditionals and extended types.

\vspace{2em}
\noindent\href{doc/html/poets/Poets-Contracts-Language-CSL-ALaCarte-Exp-Algebras-Evaluator.html}{src/Poets/Contracts/Language/CSL/ALaCarte/Exp/Algebras/Evaluator.hs} \\
\href{doc/html/poets/Poets-Contracts-Language-CSL-ALaCarte-Exp-Algebras-Marshal.html}{src/Poets/Contracts/Language/CSL/ALaCarte/Exp/Algebras/Marshal.hs} \\
\href{doc/html/poets/Poets-Contracts-Language-CSL-ALaCarte-Exp-Algebras-Render.html}{src/Poets/Contracts/Language/CSL/ALaCarte/Exp/Algebras/Render.hs} \\
\href{doc/html/poets/Poets-Contracts-Language-CSL-ALaCarte-Exp-Algebras-Substitute.html}{src/Poets/Contracts/Language/CSL/ALaCarte/Exp/Algebras/Substitute.hs} \\
\href{doc/html/poets/Poets-Contracts-Language-CSL-ALaCarte-Exp-Algebras-TypeInferer.html}{src/Poets/Contracts/Language/CSL/ALaCarte/Exp/Algebras/TypeInferer.hs} \\

These files contain the data types á la carte algebras that make out the logic of the implementation.
Among the our extensions the most notable have been in the Evaluator
algebra, which needed to be extended to use a value environment rather
than substitution, which turned out to be a somewhat involved task.

The other algebras have been extended to include support for the new
syntactic constructs.

%TODO: /src/Poets/Contracts/Language/CSL/ALaCarte/RecordFieldName/Algebras/Evaluator.hs
%TODO: /src/Poets/Contracts/Language/CSL/Analysis/ExpectedTransactions.hs


\vspace{2em}
\noindent\href{doc/html/poets/Poets-Contracts-Language-CSL-Interpreter.html}{src/Poets/Contracts/Language/CSL/Interpreter.hs}\\
\href{doc/html/poets/Poets-Contracts-Language-CSL-Parser.html}{src/Poets/Contracts/Language/CSL/Parser.hs}\\
\href{doc/html/poets/Poets-Contracts-Language-CSL-Render.html}{src/Poets/Contracts/Language/CSL/Render.hs}\\
\href{doc/html/poets/Poets-Contracts-Language-CSL-TypeChecker.html}{src/Poets/Contracts/Language/CSL/TypeChecker.hs}\\

These files contain the extensions to the interpreter, parser and type
checker, and the CSL rendering functions.


\vspace{2em}
\noindent\href{doc/html/poets/Poets-Data-ALaCarte.html}{src/Poets/Data/ALaCarte.html}
\href{doc/html/poets/Poets-Data-ALaCarte-Algebras-Evaluator.html}{src/Poets/Data/ALaCarte/Algebras/Evaluator.hs}
\href{doc/html/poets/Poets-Data-ALaCarte-Algebras-TypeInferer.html}{src/Poets/Data/ALaCarte/Algebras/TypeInferer.hs}
\href{doc/html/poets/Poets-Data-Value-Algebras-Evaluator.html}{src/Poets/Data/Value/Algebras/Evaluator.hs}

These files contain changes made to the general datatypes á la carte code.
The basic á la carte code has been supplemented with the function \verb+deepInject+,
And the Evaluator algebra has been altered to conform to the needs of having a value environment.

\vspace{2em}
\begin{verbatim}
test/contracts/Test.hs
test/contracts/test.contract
test/suites/Poets/Contracts/CSLEvaluator_Test.hs
test/suites/Poets/Contracts/CSLParser_Test.hs
test/suites/Poets/Contracts_Test.hs
test/suites/Poets_Test.hs
\end{verbatim}

These files contain the unit tests of our extensions. The files
outside the \verb+Contracts+ folder contain only administrative code
that includes the new tests to the total collection of POETS tests.

\chapter{Conclusion}
We have extended the Contract Specification Language of the POETS
platform to include primitives for writing lambda expressions,
conditional expressions, as well as defining named values which includes
functions as well as built-in list manipulation primitives. Given these
extensions, we have established a solid foundation for providing rich
functionality through the expression language, which will give more
flexibility in contract specifications.
Finally, the core features of the original language \cite[page
4]{hvitved10} have been preserved in the extended language.

%While a large portion of the extensions have been tested and verified
%the type checking is still of unknown stability, as the constraint
%solver for the CSL type inferer is not yet implemented at the time of
%writing.

\section{Future work}
As we have worked with the language we have had additional ideas to
features that could be \textit{nice to have} in the contract language.
Note that these suggestions are stricly related to the CSL expression
language, for suggestions on improvements in the greater scheme of CSL,
refer to \cite{hvitved10}.
\begin{itemize}
%\item \textbf{A module system}. 
%\item \textbf{\lstinline{let}-bindings in expressions}. (syntactic sugar for
%  lambda binding variable).
\item \textbf{Additional datatypes}. For example pairs or dynamic
  creation of new record types. To provide a richer library of primitive
  datatypes.
\item \textbf{Repeat-loop to repetitive clauses} (e.g. an action that
  has to be taken every 14 days). The purpose would be to give a better
  intuitive understanding of recursive contracts. As such, this
  extension can be implemented as purely syntactic sugar.
\item \textbf{Elimination of parameter-passing \lstinline{where}-clauses.} It is
  observed that a large amount of redundancy is performed in
  \lstinline{where}-clauses. Such an extension will be able to eliminate
  verbose operations as seen in the existing contract clauses:
  \begin{lstlisting}
  ...
  <agent> SomeAction(arguments)
  where g == goods andalso
        a == amount andalso
        c == currency
        ...
  \end{lstlisting}
  This extension will eliminate a large amount of this redundancy as
  the expressions can be collapsed into the arguments of the actions.
\end{itemize}

\defbibheading{bibliography}{\chapter{Bibliography}}
\printbibliography


\newpage
    \appendix
\chapter{Contract for partial deliveries: sale\_partial\_delivery.csl}
\label{chap:sale_partial_delivery}
\lstinputlisting{sale_partial_delivery.csl}

\end{document}
